% Options for packages loaded elsewhere
\PassOptionsToPackage{unicode}{hyperref}
\PassOptionsToPackage{hyphens}{url}
%
\documentclass[
]{article}
\usepackage{amsmath,amssymb}
\usepackage{iftex}
\ifPDFTeX
  \usepackage[T1]{fontenc}
  \usepackage[utf8]{inputenc}
  \usepackage{textcomp} % provide euro and other symbols
\else % if luatex or xetex
  \usepackage{unicode-math} % this also loads fontspec
  \defaultfontfeatures{Scale=MatchLowercase}
  \defaultfontfeatures[\rmfamily]{Ligatures=TeX,Scale=1}
\fi
\usepackage{lmodern}
\ifPDFTeX\else
  % xetex/luatex font selection
\fi
% Use upquote if available, for straight quotes in verbatim environments
\IfFileExists{upquote.sty}{\usepackage{upquote}}{}
\IfFileExists{microtype.sty}{% use microtype if available
  \usepackage[]{microtype}
  \UseMicrotypeSet[protrusion]{basicmath} % disable protrusion for tt fonts
}{}
\makeatletter
\@ifundefined{KOMAClassName}{% if non-KOMA class
  \IfFileExists{parskip.sty}{%
    \usepackage{parskip}
  }{% else
    \setlength{\parindent}{0pt}
    \setlength{\parskip}{6pt plus 2pt minus 1pt}}
}{% if KOMA class
  \KOMAoptions{parskip=half}}
\makeatother
\usepackage{xcolor}
\usepackage[margin=1in]{geometry}
\usepackage{color}
\usepackage{fancyvrb}
\newcommand{\VerbBar}{|}
\newcommand{\VERB}{\Verb[commandchars=\\\{\}]}
\DefineVerbatimEnvironment{Highlighting}{Verbatim}{commandchars=\\\{\}}
% Add ',fontsize=\small' for more characters per line
\usepackage{framed}
\definecolor{shadecolor}{RGB}{248,248,248}
\newenvironment{Shaded}{\begin{snugshade}}{\end{snugshade}}
\newcommand{\AlertTok}[1]{\textcolor[rgb]{0.94,0.16,0.16}{#1}}
\newcommand{\AnnotationTok}[1]{\textcolor[rgb]{0.56,0.35,0.01}{\textbf{\textit{#1}}}}
\newcommand{\AttributeTok}[1]{\textcolor[rgb]{0.13,0.29,0.53}{#1}}
\newcommand{\BaseNTok}[1]{\textcolor[rgb]{0.00,0.00,0.81}{#1}}
\newcommand{\BuiltInTok}[1]{#1}
\newcommand{\CharTok}[1]{\textcolor[rgb]{0.31,0.60,0.02}{#1}}
\newcommand{\CommentTok}[1]{\textcolor[rgb]{0.56,0.35,0.01}{\textit{#1}}}
\newcommand{\CommentVarTok}[1]{\textcolor[rgb]{0.56,0.35,0.01}{\textbf{\textit{#1}}}}
\newcommand{\ConstantTok}[1]{\textcolor[rgb]{0.56,0.35,0.01}{#1}}
\newcommand{\ControlFlowTok}[1]{\textcolor[rgb]{0.13,0.29,0.53}{\textbf{#1}}}
\newcommand{\DataTypeTok}[1]{\textcolor[rgb]{0.13,0.29,0.53}{#1}}
\newcommand{\DecValTok}[1]{\textcolor[rgb]{0.00,0.00,0.81}{#1}}
\newcommand{\DocumentationTok}[1]{\textcolor[rgb]{0.56,0.35,0.01}{\textbf{\textit{#1}}}}
\newcommand{\ErrorTok}[1]{\textcolor[rgb]{0.64,0.00,0.00}{\textbf{#1}}}
\newcommand{\ExtensionTok}[1]{#1}
\newcommand{\FloatTok}[1]{\textcolor[rgb]{0.00,0.00,0.81}{#1}}
\newcommand{\FunctionTok}[1]{\textcolor[rgb]{0.13,0.29,0.53}{\textbf{#1}}}
\newcommand{\ImportTok}[1]{#1}
\newcommand{\InformationTok}[1]{\textcolor[rgb]{0.56,0.35,0.01}{\textbf{\textit{#1}}}}
\newcommand{\KeywordTok}[1]{\textcolor[rgb]{0.13,0.29,0.53}{\textbf{#1}}}
\newcommand{\NormalTok}[1]{#1}
\newcommand{\OperatorTok}[1]{\textcolor[rgb]{0.81,0.36,0.00}{\textbf{#1}}}
\newcommand{\OtherTok}[1]{\textcolor[rgb]{0.56,0.35,0.01}{#1}}
\newcommand{\PreprocessorTok}[1]{\textcolor[rgb]{0.56,0.35,0.01}{\textit{#1}}}
\newcommand{\RegionMarkerTok}[1]{#1}
\newcommand{\SpecialCharTok}[1]{\textcolor[rgb]{0.81,0.36,0.00}{\textbf{#1}}}
\newcommand{\SpecialStringTok}[1]{\textcolor[rgb]{0.31,0.60,0.02}{#1}}
\newcommand{\StringTok}[1]{\textcolor[rgb]{0.31,0.60,0.02}{#1}}
\newcommand{\VariableTok}[1]{\textcolor[rgb]{0.00,0.00,0.00}{#1}}
\newcommand{\VerbatimStringTok}[1]{\textcolor[rgb]{0.31,0.60,0.02}{#1}}
\newcommand{\WarningTok}[1]{\textcolor[rgb]{0.56,0.35,0.01}{\textbf{\textit{#1}}}}
\usepackage{graphicx}
\makeatletter
\def\maxwidth{\ifdim\Gin@nat@width>\linewidth\linewidth\else\Gin@nat@width\fi}
\def\maxheight{\ifdim\Gin@nat@height>\textheight\textheight\else\Gin@nat@height\fi}
\makeatother
% Scale images if necessary, so that they will not overflow the page
% margins by default, and it is still possible to overwrite the defaults
% using explicit options in \includegraphics[width, height, ...]{}
\setkeys{Gin}{width=\maxwidth,height=\maxheight,keepaspectratio}
% Set default figure placement to htbp
\makeatletter
\def\fps@figure{htbp}
\makeatother
\setlength{\emergencystretch}{3em} % prevent overfull lines
\providecommand{\tightlist}{%
  \setlength{\itemsep}{0pt}\setlength{\parskip}{0pt}}
\setcounter{secnumdepth}{-\maxdimen} % remove section numbering
\ifLuaTeX
  \usepackage{selnolig}  % disable illegal ligatures
\fi
\IfFileExists{bookmark.sty}{\usepackage{bookmark}}{\usepackage{hyperref}}
\IfFileExists{xurl.sty}{\usepackage{xurl}}{} % add URL line breaks if available
\urlstyle{same}
\hypersetup{
  pdftitle={HW Week 7},
  hidelinks,
  pdfcreator={LaTeX via pandoc}}

\title{HW Week 7}
\author{}
\date{\vspace{-2.5em}2023-10-11}

\begin{document}
\maketitle

\hypertarget{regression-tree-model}{%
\subsection{10.1 Regression Tree model}\label{regression-tree-model}}

Let's begin by getting the data

\begin{Shaded}
\begin{Highlighting}[]
\FunctionTok{library}\NormalTok{(caret)}
\end{Highlighting}
\end{Shaded}

\begin{verbatim}
## Loading required package: ggplot2
\end{verbatim}

\begin{verbatim}
## Loading required package: lattice
\end{verbatim}

\begin{Shaded}
\begin{Highlighting}[]
\FunctionTok{library}\NormalTok{(rpart)}
\FunctionTok{library}\NormalTok{(rpart.plot)}
\FunctionTok{library}\NormalTok{(randomForest)}
\end{Highlighting}
\end{Shaded}

\begin{verbatim}
## randomForest 4.7-1.1
\end{verbatim}

\begin{verbatim}
## Type rfNews() to see new features/changes/bug fixes.
\end{verbatim}

\begin{verbatim}
## 
## Attaching package: 'randomForest'
\end{verbatim}

\begin{verbatim}
## The following object is masked from 'package:ggplot2':
## 
##     margin
\end{verbatim}

\begin{Shaded}
\begin{Highlighting}[]
\FunctionTok{library}\NormalTok{(randomForestExplainer)}
\end{Highlighting}
\end{Shaded}

\begin{verbatim}
## Registered S3 method overwritten by 'GGally':
##   method from   
##   +.gg   ggplot2
\end{verbatim}

\begin{Shaded}
\begin{Highlighting}[]
\NormalTok{myData }\OtherTok{\textless{}{-}} \FunctionTok{read.table}\NormalTok{(}\StringTok{"C:/Users/omkar/OneDrive/Documents/Analytical Tools Folder/uscrime.txt"}\NormalTok{, }\AttributeTok{header =} \ConstantTok{TRUE}\NormalTok{)}

\FunctionTok{head}\NormalTok{(myData)}
\end{Highlighting}
\end{Shaded}

\begin{verbatim}
##      M So   Ed  Po1  Po2    LF   M.F Pop   NW    U1  U2 Wealth Ineq     Prob
## 1 15.1  1  9.1  5.8  5.6 0.510  95.0  33 30.1 0.108 4.1   3940 26.1 0.084602
## 2 14.3  0 11.3 10.3  9.5 0.583 101.2  13 10.2 0.096 3.6   5570 19.4 0.029599
## 3 14.2  1  8.9  4.5  4.4 0.533  96.9  18 21.9 0.094 3.3   3180 25.0 0.083401
## 4 13.6  0 12.1 14.9 14.1 0.577  99.4 157  8.0 0.102 3.9   6730 16.7 0.015801
## 5 14.1  0 12.1 10.9 10.1 0.591  98.5  18  3.0 0.091 2.0   5780 17.4 0.041399
## 6 12.1  0 11.0 11.8 11.5 0.547  96.4  25  4.4 0.084 2.9   6890 12.6 0.034201
##      Time Crime
## 1 26.2011   791
## 2 25.2999  1635
## 3 24.3006   578
## 4 29.9012  1969
## 5 21.2998  1234
## 6 20.9995   682
\end{verbatim}

\hypertarget{including-plots}{%
\subsection{Including Plots}\label{including-plots}}

Now let's begin the process of creating a regression tree model. In week
5, we found that Ed, Po1, Ineq, M, Prob and U2 were good predictors to
use, so we will set that as the starting point.

\begin{Shaded}
\begin{Highlighting}[]
\NormalTok{ctrl }\OtherTok{\textless{}{-}} \FunctionTok{trainControl}\NormalTok{(}\AttributeTok{method =} \StringTok{"cv"}\NormalTok{, }\AttributeTok{number =} \DecValTok{5}\NormalTok{)}
\NormalTok{custom\_cp }\OtherTok{\textless{}{-}} \FunctionTok{expand.grid}\NormalTok{(}\AttributeTok{cp =} \FunctionTok{c}\NormalTok{(}\FloatTok{0.01}\NormalTok{, }\FloatTok{0.02}\NormalTok{, }\FloatTok{0.03}\NormalTok{, }\FloatTok{0.04}\NormalTok{, }\FloatTok{0.05}\NormalTok{, }\FloatTok{0.06}\NormalTok{, }\FloatTok{0.07}\NormalTok{, }\FloatTok{0.08}\NormalTok{, }\FloatTok{0.09}\NormalTok{, }\FloatTok{0.1}\NormalTok{, }\FloatTok{0.11}\NormalTok{, }\FloatTok{0.12}\NormalTok{, }\FloatTok{0.13}\NormalTok{))}
\NormalTok{model }\OtherTok{\textless{}{-}} \FunctionTok{train}\NormalTok{(Crime }\SpecialCharTok{\textasciitilde{}}\NormalTok{ Ed }\SpecialCharTok{+}\NormalTok{ Po1 }\SpecialCharTok{+}\NormalTok{ Ineq }\SpecialCharTok{+}\NormalTok{ M }\SpecialCharTok{+}\NormalTok{ Prob }\SpecialCharTok{+}\NormalTok{ U2, }\AttributeTok{data =}\NormalTok{ myData, }\AttributeTok{method =} \StringTok{"rpart"}\NormalTok{, }\AttributeTok{tuneGrid =}\NormalTok{ custom\_cp, }\AttributeTok{trControl =}\NormalTok{ ctrl)}

\NormalTok{best\_model }\OtherTok{\textless{}{-}}\NormalTok{ model}\SpecialCharTok{$}\NormalTok{finalModel}

\FunctionTok{print}\NormalTok{(model)}
\end{Highlighting}
\end{Shaded}

\begin{verbatim}
## CART 
## 
## 47 samples
##  6 predictor
## 
## No pre-processing
## Resampling: Cross-Validated (5 fold) 
## Summary of sample sizes: 38, 36, 38, 38, 38 
## Resampling results across tuning parameters:
## 
##   cp    RMSE      Rsquared   MAE     
##   0.01  338.9553  0.3601234  264.0389
##   0.02  338.9553  0.3601234  264.0389
##   0.03  338.9553  0.3601234  264.0389
##   0.04  338.9553  0.3601234  264.0389
##   0.05  338.3362  0.3611920  265.4113
##   0.06  338.3362  0.3611920  265.4113
##   0.07  338.3362  0.3611920  265.4113
##   0.08  338.3362  0.3611920  265.4113
##   0.09  346.7878  0.3242249  275.4644
##   0.10  346.7878  0.3242249  275.4644
##   0.11  346.7878  0.3242249  275.4644
##   0.12  346.7878  0.3242249  275.4644
##   0.13  348.6526  0.2944073  278.0026
## 
## RMSE was used to select the optimal model using the smallest value.
## The final value used for the model was cp = 0.08.
\end{verbatim}

When using the 5-fold cross validation, it appears that the RMSE did not
decrease significantly for cp values under 0.08. Since higher cp values
correspond to simpler trees, we will use the highest possible value,
which is 0.08. Now, let's plot the regression tree model.

\begin{Shaded}
\begin{Highlighting}[]
\FunctionTok{rpart.plot}\NormalTok{(best\_model)}
\end{Highlighting}
\end{Shaded}

\includegraphics{HW-Week-7-document_files/figure-latex/plot-1.pdf} The
interpretation of this decision tree is that if the police spending
(Po1) is less than 7.7, then the Crime prediction for the new state is
670. If Po1 is greater than 7.7, and if M is less than 13, then the
Crime Prediction is 911, else it is 1316. For this model, it appears
that only two variables are required, Po1 and M, to reach the best
possible performance.

Next, we have the random forest model.

\begin{Shaded}
\begin{Highlighting}[]
\NormalTok{rf\_model }\OtherTok{\textless{}{-}} \FunctionTok{randomForest}\NormalTok{(Crime }\SpecialCharTok{\textasciitilde{}}\NormalTok{ Ed }\SpecialCharTok{+}\NormalTok{ Po1 }\SpecialCharTok{+}\NormalTok{ Ineq }\SpecialCharTok{+}\NormalTok{ M }\SpecialCharTok{+}\NormalTok{ Prob }\SpecialCharTok{+}\NormalTok{ U2, }\AttributeTok{data =}\NormalTok{ myData, }\AttributeTok{ntree =} \DecValTok{100}\NormalTok{, }\AttributeTok{localImp =} \ConstantTok{TRUE}\NormalTok{)}
\CommentTok{\# Perform cross{-}validation}
\NormalTok{cv\_results }\OtherTok{\textless{}{-}} \FunctionTok{train}\NormalTok{(Crime }\SpecialCharTok{\textasciitilde{}}\NormalTok{ Ed }\SpecialCharTok{+}\NormalTok{ Po1 }\SpecialCharTok{+}\NormalTok{ Ineq }\SpecialCharTok{+}\NormalTok{ M }\SpecialCharTok{+}\NormalTok{ Prob }\SpecialCharTok{+}\NormalTok{ U2, }\AttributeTok{data =}\NormalTok{ myData, }\AttributeTok{method =} \StringTok{"rf"}\NormalTok{, }\AttributeTok{trControl =}\NormalTok{ ctrl)}
\FunctionTok{print}\NormalTok{(cv\_results)}
\end{Highlighting}
\end{Shaded}

\begin{verbatim}
## Random Forest 
## 
## 47 samples
##  6 predictor
## 
## No pre-processing
## Resampling: Cross-Validated (5 fold) 
## Summary of sample sizes: 36, 38, 38, 37, 39 
## Resampling results across tuning parameters:
## 
##   mtry  RMSE      Rsquared   MAE     
##   2     276.8569  0.5799343  209.9998
##   4     270.6001  0.5849774  198.2572
##   6     278.1649  0.5789581  202.1000
## 
## RMSE was used to select the optimal model using the smallest value.
## The final value used for the model was mtry = 4.
\end{verbatim}

Let's now try to explain the random forest model we have created.

\begin{Shaded}
\begin{Highlighting}[]
\NormalTok{explainer }\OtherTok{\textless{}{-}} \FunctionTok{min\_depth\_distribution}\NormalTok{(rf\_model)}
\FunctionTok{plot\_min\_depth\_distribution}\NormalTok{(explainer)}
\end{Highlighting}
\end{Shaded}

\includegraphics{HW-Week-7-document_files/figure-latex/unnamed-chunk-1-1.pdf}

\begin{Shaded}
\begin{Highlighting}[]
\CommentTok{\#print(explainer)}
\CommentTok{\#explainer \textless{}{-} explain(rf\_model, data = myData, y = myData$Crime)}
\CommentTok{\#var\_imp\_plot \textless{}{-} plot(explainer)}
\CommentTok{\#print(var\_imp\_plot)}
\end{Highlighting}
\end{Shaded}

Let's now see the importance of each variable.

\begin{Shaded}
\begin{Highlighting}[]
\NormalTok{importance }\OtherTok{\textless{}{-}} \FunctionTok{measure\_importance}\NormalTok{(rf\_model)}
\NormalTok{importance}
\end{Highlighting}
\end{Shaded}

\begin{verbatim}
##   variable mean_min_depth no_of_nodes mse_increase node_purity_increase
## 1       Ed       2.586939         196     9223.499             579687.3
## 2     Ineq       1.930510         262     2239.448             835206.7
## 3        M       2.375102         218     4158.412             520945.7
## 4      Po1       1.234694         296    70140.156            2033507.8
## 5     Prob       1.848878         256    25415.261            1272339.1
## 6       U2       2.660816         245    16996.771             572161.8
##   no_of_trees times_a_root      p_value
## 1          90           13 0.9998361852
## 2          97           16 0.1320334393
## 3          94            4 0.9762050607
## 4          98           27 0.0003175587
## 5          97           25 0.2409950663
## 6          94           15 0.5247868075
\end{verbatim}

\end{document}
